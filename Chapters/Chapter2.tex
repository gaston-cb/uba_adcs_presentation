\chapter{Introducción específica} % Main chapter title

\label{Chapter2}

%----------------------------------------------------------------------------------------
%	SECTION 2 resume 
%----------------------------------------------------------------------------------------
Se presentan las bases matemáticas del control de actitud. El control de actitud se basa en seleccionar al menos dos sistemas de referencia para definir las orientaciones a través de una matriz. Al seleccionarse dos referencias puede utilizarse diferentes parametrizaciónes de la matriz: 
\begin{itemize}
	\item Quaterniones
	\item Parámetros de Rodrigues
	\item Parámetros de Euler
	\item Parámetros de Rodrigues modificado.
	\item etc,,, 
\end{itemize}
Esta parametrización presenta ventajas sobre la matriz. La idea central del control de actitud es estimar la matriz de orientaciones. EN este contexto la matriz se llama \"matriz de Actitude\". Esta matriz se obtiene a partir de la parametrización y del modelo dinámico del sistema, donde lo que se obtiene es una estimación y no la matriz real, según el algoritmo utilizado 
 

\section{Estilo y convenciones}
\label{sec:ejemplo}

\subsection{Rotaciones activas vs Pasivas}


